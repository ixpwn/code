\documentclass{sig-alternate-10pt}
%\documentclass[letterpaper,11pt]{article}
\usepackage{url}
%\usepackage{usenix,epsfig,endnotes}
%\usepackage{fullpage} 
%\setlength{\textheight}{9in}
%\setlength{\textwidth}{6.75in}
%\setlength{\oddsidemargin}{-.125in}
\usepackage{graphicx}
%\usepackage{subfigure}
%\usepackage{ifpdf}
%\usepackage{multicol}
%\usepackage{amsmath, amssymb, amsthm}
%\usepackage{rotating}
%\onehalfspacing
\newcommand{\tbd}[1]{[{\bf{#1}}]}
%\newcommand{\tbd}[1]{}
\newcommand{\ie}{{\it i.e.}}
\newcommand{\eg}{{\it e.g.}}
\newcommand{\eat}[1]{}

%\newcommand{\tbd}[1]{}
%\setlength\topmargin{0in}
%\usepackage{verbatim}
%\usepackage[compact]{titlesec}
%\usepackage[small]{caption}
\usepackage{times}
%\titlespacing{\section}{0pt}{*0}{*0}
%\titlespacing{\subsection}{0pt}{*0}{*0}
%\titlespacing{\subsubsection}{0pt}{*0}{*0}
%\titlespacing{\abstract}{0pt}{*0}{*0}
%%REPLACES ITEMIZE

\title{A Glance at Physical Connectivity and Internet Resilience}
\author{\begin{tabular}{ccc} Shaddi Hasan & Justine Sherry & Kristin Stephens\\
%        \begin{multicols}{2}{{\it Draft - Please do not distribute.}}\\
        \end{tabular}\\
        University of California, Berkeley\titlenote{Authors listed alphabetically.}\\\\
        \large
        {\it Draft: Please Do Not Distribute}}
        \normalsize
%\author{Paper \#69, 14 Pages}
\date{}
\begin{document}	
    \maketitle

    \abstract{\it
        Internet connectivity is an integral component of modern commerce and communication.
        Loss of Internet connectivity can shut down industries, limit the spread of information, and undermine free speech.
        However, physical bottlenecks in the Internet's structure leave it in many cases vulnerable to disconnection.
        In this paper, we investigate where these physical bottlenecks lie, leaving network communucation suceptible to interruption in cases of disaster or direct attack.
        Our analysis estimates that\ldots
     }
    
    \section{Introduction}
        The United States government defines {\it critical infrastructure} as ``systems
and assets, whether physical or virtual, so vital to the United States that the
incapacity or destruction of such systems and assets would have a debilitation
impact on security, national economic security, national public health or
safety, or any combination of those matters''~\cite{patriotact}. Internet
connectivity, a fundamental requirement for modern communication, is
undoubtedly such a system.  Businesses~\cite{something?} and
governments~\cite{cyberspacepolicy} hence value the physical security of
Internet infrastructure to disaster or attack.


In this paper, we provide a brief glance at physical Internet resilience by
focusing on a model of geographic locations where multiple networks converge to
interconnect. Damage to an Internet Exchange Point (IXP) or Point-of-Presence
(PoP) are worst-case scenarios for physical disaster, as these locations
provide connectivity between anywhere from two to hundreds of networks.
Building on prior work which develops techniques for identifying formal
IXPs~\cite{ixps-mapped}, discovering PoPs in general~\cite{iplane}, and
pinpointing border routers~\cite{asbrsomething}, we develop an AS connectivity
graph that focuses on the locations where peering takes place.  We then
consider the behavior of an adversary who seeks to maximize damage to this
connectivity graph.  Such an adversary might wish to disconnect two Tier 1
networks, isolate a large number of networks, or partition a geographic region
from the rest of the Internet.  We consider each in turn.


We are not the first to evaluate Internet resilience to
failures~\cite{michigan, measuringresilience, resilience-under-BGP,
resilience-complex-networks}.  Our contribution is to consider physical
locations of conncetivity as {\it failure points}.  Previous work focused on
the impact of severing logical links, that is, declaring that two networks had
been entirely disconnected.  This overestimates realistic damage in some cases,
and underestimates in others.  For example, it is highly unlikely that two Tier
1 networks would be partitioned in any single event.  These networks have
global footprints and connect at multiple physical locations.  Thus,
disconnecting Tier 1 networks is an overestimate of damage in all scenarios
except a coordinated attack.  On the other hand, disconnecting logical links
one by one ignores the physical reality that any disaster that strikes a link
in shared infrastructure such as an IXP is likely to impact a large number of
links.  In this case, modeling failure of a single logical link is an
underestimate.  We argue that focusing on physical connectivity points is a
more realistic model of the impact of physical disaster. 

\eat{ Not true!
 
Second, we consider resilience in an adversarial setting.  Recent events in
Egypt~\cite{thenews} and other countries involved government intervention to
sever Internet access in order to limit communication between revolutionaries
and the outside world.  Further, the criticality of Internet communication
makes it a prime target for terrorist attack.  Attacks by any human entity may
involve damage to one or more physical locations.  Hence, our analysis is a
superset of previous work which focused primarily on natural disaster or
sideffects from events in a single location.
}

Before we move forward, we clarify the scope of our goals.  
Evaluations of AS-level connectivity graphs show that traceroute-based
topologies like those that we rely upon are in many ways
incomplete~\cite{walter}.  For instance, policy compliant routing ensures that
measurements made with only limited vantage points cannot observe peering
relationships between networks where neither network contains a measurement
vantage point.  Further, `backup links', which are provisioned for the case of
failure but otherwise unused, cannot be observed since no traffic flows across
these links when the primary links are available.  These and other impediments
mean that any analysis over measurement-based graphs may be missing critical
information.  Thus, the results of our study should not be considered hard and
fast projections of Internet connectivity.  Instead, we aim only to provide
{\it estimates} and {\it bounds} on the impact of attack on Internet
infrastructure.

Our results estimate that \ldots \justine{My hypotheses: (1) we find that
Tier-1's are too well connected to feasibly partition - we can echo the
argument from that paper Kristin found that the threat is misconfiguration or
attacks on the control plane, but certainly not the physical infrastructure.
(2) The exact opposite is true for small networks - that taking out a single
IXP in many cases is enough to knock offline dozens of ASes and hundreds or
thousands of prefixes. WRT geographic regions, I am less sure - hopefully we
can identify the Egypts of the world?}

The remainder of the paper is organized as follows.
Section~\ref{sec:connectivity} discusses our dataset and model of the physical
connectivity of Internet infrastructure.  Section~\ref{sec:algos} provides the
algorithms we used in analyzing this model.  Section~\ref{sec:results}
describes our discoveries from applying these algorithms.  Finally, we discuss
related work in \S~\ref{sec:relatedwork} and conclude in
\S~\ref{sec:conclusion}.



    \section{Connectivity Model}
        What our data is.
        How we piece it together.
        \subsubsection*{Evaluation of Model}
            \begin{itemize}
                \item survey random network providers and ask them if what we found was correct
                \item compare physical links discovered to logical connectivity graph (CAIDA provides this) - what fraction of logical links did we observe? For Tier-1's? For Tier-2's? For stub networks?
            \end{itemize} 

    \section{Algorithmic Analysis}
        How we do:
        \begin{enumerate}
            \item Easy peasy: partition Tier-1's
            \item Maximize number of nodes that are completely isolated
            \item Focus partition to a geographic region
        \end{enumerate}
        
    \section{Results}
        Graphs and stuff.
        \subsubsection*{Comparison to Recent Events}
        \tbd{discuss Egypt and Iran?}
    \section{Related Work}
	Our work builds on and takes inspiration from research in Internet resilience analysis, experience from Internet outages, PoP-level Internet topology measurement studies, and network infrastructure security policy.

{\bf Resilience Analysis.}
    Comprehensive analysis of Internet resilience remains beyond reach due to limited access to global routing and topology data. 
    However, several limited studies have provided insight into the impact of logical link failures and opportunities to make the AS graph more robust to these failures.
    
    Wu et. al~\cite{michigan} provide the most comprehensive analysis of Internet resilience under {\it logical} link failure.
    Using an AS-level graph, they removed one or more peering relationships and evaluated the availability of policy-compliant paths between impacted networks before and after the logical link failure.
    
    \begin{itemize}
        \item measuring~\cite{measuringresilience}
    \end{itemize}
    Hu et. al~\cite{ixp-routingdiversity}
{\bf Outage Events.}
\begin{itemize}
    \item Restoration study~\cite{taiwan}
\end{itemize}

{\bf PoP-level Internet Topologies.}
    Our model for network connectivity focuses on {\it failure points}, physical locations where multiple networks connect, leading to multiple correlated failures in case of disaster.
    We borrow techniques and data for discovering these physical locations from iPlane~\cite{iplane} and the IXP Mapping Project~\cite{ixps-mapped}.
    iPlane clusters IP addresses into PoPs using a combination of DNS-based geolocation and TTL-based distance measurements.
    The IXP Mapping project uses public IXP membership datasets, DNS names, looking glass servers, BGP tables, active traceroute and ping measurements, and other sources to provide the most accurate IXP membership datasets available to date. 

{\bf Network Security Policy.}
    \begin{itemize}
        \item White House POlicy Review~\cite{cyberspacepolicy} 
    \end{itemize}


    \section{Conclusion}
        \tbd{restate what we did and what we found}
        \tbd{our conclusions: network operators need not just logical but physical redundancy - diversity of location. also, tradeoff between efficient exchange in IXPs (getting lots of people to same place) and creating a juicy target. thus we should really really secure these places!}
%\scriptsize
\bibliographystyle{abbrv}
\bibliography{cite}

%\input{appendix}

\end{document}
