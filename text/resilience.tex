\documentclass{sig-alternate-10pt}
%\documentclass[letterpaper,11pt]{article}
\usepackage{url}
%\usepackage{usenix,epsfig,endnotes}
%\usepackage{fullpage} 
%\setlength{\textheight}{9in}
%\setlength{\textwidth}{6.75in}
%\setlength{\oddsidemargin}{-.125in}
\usepackage{graphicx}
%\usepackage{subfigure}
%\usepackage{ifpdf}
%\usepackage{multicol}
%\usepackage{amsmath, amssymb, amsthm}
%\usepackage{rotating}
%\onehalfspacing
\newcommand{\tbd}[1]{[{\bf{#1}}]}

\newcommand{\justine}[1]{[{\bf justine: {#1}}]}
\newcommand{\shaddi}[1]{[{\bf shaddi: {#1}}]}
%\newcommand{\tbd}[1]{}
\newcommand{\ie}{{\it i.e.}}
\newcommand{\eg}{{\it e.g.}}
\newcommand{\eat}[1]{}

%\newcommand{\tbd}[1]{}
%\setlength\topmargin{0in}
%\usepackage{verbatim}
%\usepackage[compact]{titlesec}
%\usepackage[small]{caption}
\usepackage{times}
%\titlespacing{\section}{0pt}{*0}{*0}
%\titlespacing{\subsection}{0pt}{*0}{*0}
%\titlespacing{\subsubsection}{0pt}{*0}{*0}
%\titlespacing{\abstract}{0pt}{*0}{*0}
%%REPLACES ITEMIZE

\title{A Glance at Physical Connectivity and Internet Resilience}
\author{\begin{tabular}{ccc} Shaddi Hasan & Justine Sherry & Kristin Stephens\\
%        \begin{multicols}{2}{{\it Draft - Please do not distribute.}}\\
        \end{tabular}\\
	University of California, Berkeley\titlenote{Authors listed
	alphabetically.}\\\\
        \large
        {\it Draft: Please Do Not Distribute}}
        \normalsize
%\author{Paper \#69, 14 Pages}
\date{}
\begin{document}	
    \maketitle

    \abstract{\it
	Internet connectivity is an integral component of modern commerce and
	communication.  Loss of Internet connectivity can shut down industries,
	limit the spread of information, and undermine free speech.  However,
	physical bottlenecks in the Internet's structure leave it in many cases
	vulnerable to disconnection.  In this paper, we investigate where these
	physical bottlenecks lie, leaving network communication susceptible to
	interruption in cases of disaster or direct attack.  Our analysis
	estimates that\ldots
     }
    
    \section{Introduction}
        Earthquakes, hurricanes, and other large-scale disasters cause widespread damage
to physical infrastructure, and Internet routing infrastructure is no exception.
In 2006, an earthquake in Taiwan severed six of seven 
undersea cables connecting North and Southeast Asia with each other and North 
America. 
Regional disasters like these have the potential to generate substantial 
correlated failures when multiple links disconnect and entire Exchange Points
go dark.

\eat{
The United States government defines {\it critical infrastructure} as ``systems
and assets, whether physical or virtual, so vital to the United States that the
incapacity or destruction of such systems and assets would have a debilitation
impact on security, national economic security, national public health or
safety, or any combination of those matters''~\cite{patriotact}. Internet
connectivity, a fundamental requirement for modern communication, is
undoubtedly such a system.  Businesses~\cite{something?} and
governments~\cite{cyberspacepolicy} hence value the physical security of
Internet infrastructure to disaster.
}

In this paper, we explore the impact of regional disasters on Internet
connectivity using a new modeling technique, geodesic failure regionalization.
While previous work in Internet resilience focused primarily on logical
connectivity, we map peering relationships to one or more geographic regions
where peering takes place.  We approximate geographic regions by generating a
polyhedron whose faces represent a contiguous area suceptible to disaster.  We
then use publicly available peering datasets~\cite{brice, peeringdb},
traceroutes, and DNS and commercial geolocation techniques to map locations
where networks peer (exchange points or private peering facilities) to faces on
the polyhedron.  We use this model to simulate the impact of regional failures,
\ie{} the removal of one or more faces, and investigate the following issues:

\noindent{\bf Regional Peering Redundancy.} How many regions do networks
typically peer in? How does it scale with the degree of the network?

\noindent{\bf Quantifying Resilience.} Can we quantify the resiliance of a
network to failure terms of it's number of peers, global footprint, and number
of regions it peers in?

\noindent{\bf Identifying Worst-Case Scenarios.} Are there bottlenecks in the
physical connectivity graph? How badly could the network partition in a
worst-case regional disaster?

We are not the first to evaluate Internet resilience to
failures~\cite{michigan, measuringresilience, resilience-under-BGP,
resilience-complex-networks}.  Our contribution is to consider the impact of
physical disaster on the observed connectivity map.  Previous work focused on
the impact of severing logical links, that is, declaring that two networks had
been entirely disconnected.  This type of model cannot capture the network
impact of physical disaster.  For example, it is highly unlikely that two Tier
1 networks would be partitioned in any single event.  These networks have
global footprints and connect at multiple physical locations.  On the other
hand, disconnecting logical links one by one ignores the physical reality that
any disaster that strikes a link in shared infrastructure such as an IXP is
likely to impact a large number of links.  In this case, modeling failure of a
single logical link is an underestimate.  We argue that focusing on physical
connectivity is the best way to develop models of the impact of physical
disaster. 


Before we move forward, we clarify the scope of our goals.  Evaluations of
AS-level connectivity graphs show that traceroute-based topologies like those
that we rely upon are in many ways incomplete~\cite{walter}.  For instance,
policy compliant routing ensures that measurements made with only limited
vantage points cannot observe peering relationships between networks where
neither network contains a measurement vantage point.  Further, `backup links',
which are provisioned for the case of failure but otherwise unused, cannot be
observed since no traffic flows across these links when the primary links are
available.  These and other impediments mean that any analysis over
measurement-based graphs may be missing critical information.  Thus, the
results of our study should not be considered hard and fast projections of
Internet connectivity.  Instead, we aim only to provide {\it estimates} and
{\it bounds} on the impact of regional disaster on Internet routing.

Our results estimate that \justine{\ldots}

The remainder of the paper is organized as follows \justine{\ldots}:.



    \section{Connectivity Model}
        What our data is.
        How we piece it together.
        \subsubsection*{Evaluation of Model}
            \begin{itemize}
		\item survey random network providers and ask them if what we
		found was correct
		\item compare physical links discovered to logical connectivity
		graph (CAIDA provides this) - what fraction of logical links
		did we observe? For Tier-1's? For Tier-2's? For stub networks?
            \end{itemize} 

    \section{Algorithmic Analysis}
        How we do:
        \begin{enumerate}
            \item Easy peasy: partition Tier-1's
            \item Maximize number of nodes that are completely isolated
            \item Focus partition to a geographic region
        \end{enumerate}
        
    \section{Results}
        Graphs and stuff.
        \subsubsection*{Comparison to Recent Events}
        \tbd{discuss Egypt and Iran?}
    \section{Related Work}
	Our work builds on and takes inspiration from research in Internet resilience analysis, experience from Internet outages, PoP-level Internet topology measurement studies, and network infrastructure security policy.

{\bf Resilience Analysis.}
    Comprehensive analysis of Internet resilience remains beyond reach due to limited access to global routing and topology data. 
    However, several limited studies have provided insight into the impact of logical link failures and opportunities to make the AS graph more robust to these failures.
    
    Wu et. al~\cite{michigan} provide the most in-depth analysis of Internet resilience under {\it logical} link failure.
    Using an AS-level graph, they removed one or more peering relationships and evaluated the availability of policy-compliant paths between impacted networks before and after the logical link failure.

    Omer et. al~\cite{measuringresilience} consider undersea cables as capacity bottlenecks for Internet connectivity between continents. 
    Like us, they look for constraints in the physical, rather than logical connectivity graph. 
    However, their model looks at a courser level of connectivity and does not focus on IXPs as correlated failure points. 
    
    Hu et. al~\cite{ixp-routingdiversity} also investigate logical failures on an AS-level graph, focusing opportunities to limit the impact of logical failures.
    %They looked at relaxing peering constraints (relax valley-freeness)
    Like us, they augmented thair AS graph with IXP connectivity data.
    However, they did not use this data to consider the possibility of correlated failures.
    Rather, they investigated using IXPs to provide backup peers to improve connectivity in case of emergency. 

{\bf Outage Events.}
\begin{itemize}
    \item Restoration study~\cite{taiwan}
\end{itemize}

{\bf PoP-level Internet Topologies.}
    Our model for network connectivity focuses on {\it failure points}, physical locations where multiple networks connect, leading to multiple correlated failures in case of disaster.
    We borrow techniques and data for discovering these physical locations from iPlane~\cite{iplane} and the IXP Mapping Project~\cite{ixps-mapped}.
    iPlane clusters IP addresses into PoPs using a combination of DNS-based geolocation and TTL-based distance measurements.
    The IXP Mapping project uses public IXP membership datasets, DNS names, looking glass servers, BGP tables, active traceroute and ping measurements, and other sources to provide the most accurate IXP membership datasets available to date. 

{\bf Network Security Policy.}
    \begin{itemize}
        \item White House POlicy Review~\cite{cyberspacepolicy} 
    \end{itemize}


    \section{Conclusion}
        \tbd{restate what we did and what we found}
	\tbd{our conclusions: network operators need not just logical but
	physical redundancy - diversity of location. also, tradeoff between
	efficient exchange in IXPs (getting lots of people to same place) and
	creating a juicy target. thus we should really really secure these
	places!}
%\scriptsize
\bibliographystyle{abbrv}
\bibliography{cite}

%\input{appendix}

\end{document}
