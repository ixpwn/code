\documentclass{sig-alternate-10pt}
%\documentclass[letterpaper,11pt]{article}
\usepackage{url}
%\usepackage{usenix,epsfig,endnotes}
%\usepackage{fullpage} 
%\setlength{\textheight}{9in}
%\setlength{\textwidth}{6.75in}
%\setlength{\oddsidemargin}{-.125in}
\usepackage{graphicx}
%\usepackage{subfigure}
%\usepackage{ifpdf}
%\usepackage{multicol}
%\usepackage{amsmath, amssymb, amsthm}
%\usepackage{rotating}
\usepackage{multirow}
\usepackage{rotating}
%\onehalfspacing

%COMMENT FOR FINAL VERSION
\newcommand{\tbd}[1]{[{\bf{#1}}]}
\newcommand{\justine}[1]{[{\bf justine: {#1}}]}
\newcommand{\shaddi}[1]{[{\bf shaddi: {#1}}]}
\newcommand{\kristin}[1]{[{\bf kristin: {#1}}]}

%UNCOMMENT FOR FINAL VERSION
%\newcommand{\tbd}[1]{}
%\newcommand{\justine}[1]{}
%\newcommand{\shaddi}[1]{}
%\newcommand{\kristin}[1]{}


\newcommand{\ie}{{\it i.e.}}
\newcommand{\eg}{{\it e.g.}}
\newcommand{\eat}[1]{}

%\newcommand{\tbd}[1]{}
%\setlength\topmargin{0in}
%\usepackage{verbatim}
%\usepackage[compact]{titlesec}
%\usepackage[small]{caption}
\usepackage{times}
%\titlespacing{\section}{0pt}{*0}{*0}
%\titlespacing{\subsection}{0pt}{*0}{*0}
%\titlespacing{\subsubsection}{0pt}{*0}{*0}
%\titlespacing{\abstract}{0pt}{*0}{*0}
%%REPLACES ITEMIZE

\title{Modeling Regional Failures in Interdomain Connectivity}
\author{\begin{tabular}{ccc} Shaddi Hasan & Justine Sherry & Kristin Stephens\\
%        \begin{multicols}{2}{{\it Draft - Please do not distribute.}}\\
        \end{tabular}\\
    University of California, Berkeley\titlenote{Authors listed
    alphabetically.}\\\\
        \large
        {\it CS270 Final Project}}
        \normalsize
%\author{Paper \#69, 14 Pages}
\date{}
\begin{document}    
    \maketitle

        %Internet interdomain resilience requires engineering for failure, from
        %deploying robust routing protocols, to establishing logical
        %connectivity redundancy (multihoming), to building redundant
        %interomain physical links.  
    \abstract{\it
        Earthquakes, hurricanes, and other regional disasters can 
        stress Internet interdomain connectivity.  Because of their widespread
        impact, these disasters can cause failures of entire exchange points or
        severance of long distance cables, leading to correlated failures in
        interdomain connectivity.  In this paper we develop a model, geodesic
        failure regionalization, to explore the impact of regional failures on
        AS connectivity.  We map logical AS connectivity links to geographic
        regions where peering occurs to model the regional redundancy of
        peering connections.
        By modeling the redundancy of logical AS
        connectivity, we can perform
        finer-grained analysis of the impact of catastrophe than work which
        relies only on logical AS connectivity alone.  
        Our data demonstrates that most transit-providing networks peer in redundant locations, such that small physical disasters ar unlikely to partition two peering networks.    
    However, our data also includes geographic regions served by only a limited number of ASes with limited peering to external networks, leaving these geographic regions suceptible to disconnection after damage to only a few regions containing all of their egress connectivity.
    }
    
    \section{Introduction}
        Earthquakes, hurricanes, and other large-scale disasters cause widespread damage
to physical infrastructure, and Internet routing infrastructure is no exception.
In 2006, an earthquake in Taiwan severed six of seven 
undersea cables connecting North and Southeast Asia with each other and North 
America. 
Regional disasters like these have the potential to generate substantial 
correlated failures when multiple links disconnect and entire Exchange Points
go dark.

\eat{
The United States government defines {\it critical infrastructure} as ``systems
and assets, whether physical or virtual, so vital to the United States that the
incapacity or destruction of such systems and assets would have a debilitation
impact on security, national economic security, national public health or
safety, or any combination of those matters''~\cite{patriotact}. Internet
connectivity, a fundamental requirement for modern communication, is
undoubtedly such a system.  Businesses~\cite{something?} and
governments~\cite{cyberspacepolicy} hence value the physical security of
Internet infrastructure to disaster.
}

In this paper, we explore the impact of regional disasters on Internet
connectivity using a new modeling technique, geodesic failure regionalization.
While previous work in Internet resilience focused primarily on logical
connectivity, we map peering relationships to one or more geographic regions
where peering takes place.  We approximate geographic regions by generating a
polyhedron whose faces represent a contiguous area suceptible to disaster.  We
then use publicly available peering datasets~\cite{brice, peeringdb},
traceroutes, and DNS and commercial geolocation techniques to map locations
where networks peer (exchange points or private peering facilities) to faces on
the polyhedron.  We use this model to simulate the impact of regional failures,
\ie{} the removal of one or more faces, and investigate the following issues:

\noindent{\bf Regional Peering Redundancy.} How many regions do networks
typically peer in? How does it scale with the degree of the network?

\noindent{\bf Quantifying Resilience.} Can we quantify the resiliance of a
network to failure terms of it's number of peers, global footprint, and number
of regions it peers in?

\noindent{\bf Identifying Worst-Case Scenarios.} Are there bottlenecks in the
physical connectivity graph? How badly could the network partition in a
worst-case regional disaster?

We are not the first to evaluate Internet resilience to
failures~\cite{michigan, measuringresilience, resilience-under-BGP,
resilience-complex-networks}.  Our contribution is to consider the impact of
physical disaster on the observed connectivity map.  Previous work focused on
the impact of severing logical links, that is, declaring that two networks had
been entirely disconnected.  This type of model cannot capture the network
impact of physical disaster.  For example, it is highly unlikely that two Tier
1 networks would be partitioned in any single event.  These networks have
global footprints and connect at multiple physical locations.  On the other
hand, disconnecting logical links one by one ignores the physical reality that
any disaster that strikes a link in shared infrastructure such as an IXP is
likely to impact a large number of links.  In this case, modeling failure of a
single logical link is an underestimate.  We argue that focusing on physical
connectivity is the best way to develop models of the impact of physical
disaster. 


Before we move forward, we clarify the scope of our goals.  Evaluations of
AS-level connectivity graphs show that traceroute-based topologies like those
that we rely upon are in many ways incomplete~\cite{walter}.  For instance,
policy compliant routing ensures that measurements made with only limited
vantage points cannot observe peering relationships between networks where
neither network contains a measurement vantage point.  Further, `backup links',
which are provisioned for the case of failure but otherwise unused, cannot be
observed since no traffic flows across these links when the primary links are
available.  These and other impediments mean that any analysis over
measurement-based graphs may be missing critical information.  Thus, the
results of our study should not be considered hard and fast projections of
Internet connectivity.  Instead, we aim only to provide {\it estimates} and
{\it bounds} on the impact of regional disaster on Internet routing.

Our results estimate that \justine{\ldots}

The remainder of the paper is organized as follows \justine{\ldots}:.



    \section{Connectivity Model}
        \label{sec:connectivity_model}
            To perform our analysis of network-level, geographic, and service isolation in face of large-scale physical disaster, we generated a model of AS connectivity which incorporates physical locations of AS peering, and maps ASes to geographic regions they serve. 
    We develop our dataset by borrowing techniques and data from
    the IXP
    Mapping Project~\cite{ixps-mapped}, iPlane~\cite{iplane},
    Aqualab~\cite{sidewalk},
    CAIDA~\cite{caidadata}, and the MaxMind~\cite{maxmind} and
IP2Location~\cite{ip2loc} industrial IP geolocation services.

    To model connectivity, we generate an AS-level graph with hyperedges for
each PoP or {\it failure point} where ASes connect.
    Thus, a pair of ASes may be joined by multiple edges if they peer in
redundant physical locations.
    Further, a single hyperedge may join two or more ASes when many ASes
converge on a the same location for peering.
    \justine{figure?}
    
    We use a measurement-based technique borrowed from iPlane~\cite{iplane}
generate a graph representing real world peering arrangements.  
    Our starting dataset was a set of traceroutes issued through June and July 2010 by
iPlane~\cite{iplane} and Aqualab~\cite{aqualab}, and CAIDA's~\cite{caida} July
2010 Internet Topology Datakit.
    We then mapped IP addresses which appeared adjacent to eachother in the
router-level topology to their AS numbers. 
    We collected all IP addresses which we identified as `border routers', that
is, that connected to an IP address which we identified as belonging to another
AS.
    We issued UDP pings to each of the border routers from 134
PlanetLab~\cite{planetlab} nodes and used the TTL values from the replies to
generate estimated reverse path lengths for each node. 
    We also geolocated each IP address using a combination of the
MaxMind~\cite{maxmind} and IP2Location~\cite{ip2loc} industrial geolocation
services.
    Using these TTL values and geolocated latitudes and longitudes, we used
a modified version of iPlane's~\cite{iplane} PoP-detection algorithm.
    The technique generates clusters based on both geolocation and similar
TTL-values. 
    However, where iPlane creates PoPs which only contain IP addresses from a
single AS, we allowed multiple ASes to be clustered into the same PoP. 
    For each resulting PoP, we created a hyperedge including all of the ASes
with IP addresses within the PoP, and annotated it with the assigned geographic
location.

    The resulting graph contained {\bf n} PoPs with an average of {\bf m}
ASes... \justine{describe dataset}. 

    We then annotated each AS with a geographic `coverage' region, to provide
an estimate of the area served by each network. 
    To do this, we sampled \justine{a fraction} of each network's IP space and
once again geolocated each address. 
    We then used our sampling to generate a (latitude, longitude) bounding box
around the region served.    
    \justine{describe dataset}.
    \justine{refer to figure}.
    \justine{For class project: let's calculate some Chernoff bounds so that we
can be smart about our sample, given that MaxMind estimates 83\% accuracy?}
 
        \subsubsection*{Evaluation of Model}
            \begin{itemize}
        \item survey random network providers and ask them if what we
        found was correct
        \item Compare iPLaney data to Brice (Manual) Data.
        \item compare physical links discovered to logical connectivity
        graph (CAIDA provides this) - what fraction of logical links
        did we observe? For Tier-1's? For Tier-2's? For stub networks?
            \end{itemize} 

\begin{figure}[tb]
\centering
\includegraphics[width=3.25in]{graph_all_match_ccdf}
\caption[]{For ground truth peering locations, CCDF of distance to closest AS adjacency in our dataset using varying geolocation techniques. X-axis is in log scale.} 
%The $y$-axis is fraction of AS pairs with a path between them that traverses a service supporting AS. The $x$-axis is the fraction of ASes in the simulation topology supporting the service. }
%Using MIRO~\cite{miro} style multipath allows networks to provide access to services 
%even when their default path does not encounter a service-supporting AS.}
\end{figure}




    \section{Measuring Connectivity}
        \label{sec:quant_connect}
        \begin{figure}[tb]
\centering
\includegraphics[width=3.25in]{scatter}
\caption[]{For ground truth peering locations, CCDF of distance to closest AS
adjacency in our dataset using varying geolocation techniques. X-axis is in log
scale.} 
%Using MIRO~\cite{miro} style multipath allows networks to provide access to
%services even when their default path does not encounter a service-supporting
%AS.}
\end{figure}


\begin{figure}[tb]
\centering
\includegraphics[width=3.25in]{peering}
\caption[]{For ground truth peering locations, CCDF of distance to closest AS
adjacency in our dataset using varying geolocation techniques. X-axis is in log
scale.} 
%Using MIRO~\cite{miro} style multipath allows networks to provide access to
%services even when their default path does not encounter a service-supporting
%AS.}
\end{figure}




    \section{Regional Failure Analysis}
        \label{sec:failures} 
            \begin{table}
        \centering
        \begin{tabular}{lc|l|l|l|l|l|l}
            &&\multicolumn{5}{c}{\bf Number of Partitions}\\
            &&{\bf 2}&{\bf 4}&{\bf 8}&{\bf 16}&{\bf 32}\\
            \hline
            \multirow{5}{*}{\begin{sideways}{\bf Imbalance (\%)}\end{sideways}}
            &{\bf 5}&1036&1520&1818&1993&2087\\
            &{\bf 10}&1045&1495&1809&1987&2085\\
            &{\bf 15}&1031&1481&1798&1981&2097\\
            &{\bf 20}&1000&1482&1787&1971&2090\\
            &{\bf 25}&986&1464&1777&1965&2085\\
        \end{tabular}
        \caption[]{\label{tbl:hmetis} Number of cells to remove in order to fully partition the network, such that no users from one partition can communicate with any user in another partition. Columns indicate the number of desired partitions, and rows indicate the allowed imbalance between the largest partition and the average size (so for imbalance 5\%, k partitions, and n cells a partition cannot be more than $1.05 * \frac{n}{k}$). \kristin{fill in numbers}}
    \end{table}
        
    Turning to our final goal, analysis of worst-case scenarios, we attend to two questions. 
    First, how much physical disaster would be required to fully partition the network?         
    Second, are there regional bottlenecks in the physical connectivity graph?

    \subsubsection*{Network Partitioning}
    We whimsically address an action-movie style disaster with the first question: how many physical disasters would it take for half of all prefixes to be in one partition, and half of all prefixes to be in another, such that no users in one partition can speak to the users in another partition?
    While the question is somewhat outrageous, the answer illuminates the redundancy of geographic connectivity accross the globe, as we find that minimum cuts across our regional connectivity graph are quite large.
    
    To address this question, we use the {\tt hmetis} software package~\cite{hmetis} to discover balanced graph cuts.
    {\tt hmetis} uses \justine{technique...?} to discover minimum cuts that leave behind a balanced number of vertices on either side of the cut.
    \justine{what do we feed in to hmetis. need to weight by pfx.}     
 
    \subsubsection*{Regional Bottlenecks}
    We now turn to a more practical question of recent political interest: regional bottlenecks in connectivity.
    Recent events in Iran~\cite{iran}, Egypt~\cite{egypt}, and China~\cite{china} exemplify the implcations of the fact that each nation's connectivity to the outside world traverses one or a handfull of egress points.
    This leaves the countries vulnerable to malicious or accidental failure at these bottlenecks.
    We seek to identify regions which are similarly vulnerable, by once again examining our regional connectivity graph. 

        
    \section{Related Work}
        \label{sec:related_work}
        Our work builds on and takes inspiration from research in Internet resilience analysis, experience from Internet outages, PoP-level Internet topology measurement studies, and network infrastructure security policy.

{\bf Resilience Analysis.}
    Comprehensive analysis of Internet resilience remains beyond reach due to limited access to global routing and topology data. 
    However, several limited studies have provided insight into the impact of logical link failures and opportunities to make the AS graph more robust to these failures.
    
    Wu et. al~\cite{michigan} provide the most in-depth analysis of Internet resilience under {\it logical} link failure.
    Using an AS-level graph, they removed one or more peering relationships and evaluated the availability of policy-compliant paths between impacted networks before and after the logical link failure.

    Omer et. al~\cite{measuringresilience} consider undersea cables as capacity bottlenecks for Internet connectivity between continents. 
    Like us, they look for constraints in the physical, rather than logical connectivity graph. 
    However, their model looks at a courser level of connectivity and does not focus on IXPs as correlated failure points. 
    
    Hu et. al~\cite{ixp-routingdiversity} also investigate logical failures on an AS-level graph, focusing opportunities to limit the impact of logical failures.
    %They looked at relaxing peering constraints (relax valley-freeness)
    Like us, they augmented thair AS graph with IXP connectivity data.
    However, they did not use this data to consider the possibility of correlated failures.
    Rather, they investigated using IXPs to provide backup peers to improve connectivity in case of emergency. 

{\bf Outage Events.}
\begin{itemize}
    \item Restoration study~\cite{taiwan}
\end{itemize}

{\bf PoP-level Internet Topologies.}
    Our model for network connectivity focuses on {\it failure points}, physical locations where multiple networks connect, leading to multiple correlated failures in case of disaster.
    We borrow techniques and data for discovering these physical locations from iPlane~\cite{iplane} and the IXP Mapping Project~\cite{ixps-mapped}.
    iPlane clusters IP addresses into PoPs using a combination of DNS-based geolocation and TTL-based distance measurements.
    The IXP Mapping project uses public IXP membership datasets, DNS names, looking glass servers, BGP tables, active traceroute and ping measurements, and other sources to provide the most accurate IXP membership datasets available to date. 

{\bf Network Security Policy.}
    \begin{itemize}
        \item White House POlicy Review~\cite{cyberspacepolicy} 
    \end{itemize}


    \section{Conclusion}
        \label{sec:conclusion}    
            We performed a geography-based analysis of Internet resilience to large-scale disaster, such as earthquake or hurricane.
    Our technique, geodesic failure regionalization, partitions the globe into hexagonal regions and maps AS adjacencies to fate-sharing `failure regions.'
    We observed that most AS-level peering between transit networks is redundant, occurring in two or more cell regions.
    We also showed that the Internet's physical topology graph as a whole is strongly connected, with a minimum of 986 regions requiring complete obliteration to fully halve the Internet, with half of all prefixes in a partition leaving them fully unable to communicate with any prefix in the other partition.
    However, regional partitioning remains a threat, as less well-connected parts of the world direct most of their traffic through a limited number of peering regions. \shaddi{examples}



%\scriptsize
%\small
\bibliographystyle{abbrv}
\bibliography{cite}

%\input{appendix}

\end{document}
