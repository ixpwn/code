    We performed a geography-based analysis of Internet resilience to large-scale disaster, such as earthquake or hurricane.
    Our technique, geodesic failure regionalization, partitions the globe into hexagonal regions and maps AS adjacencies to fate-sharing `failure regions.'
    We observed that most AS-level peering between transit networks is redundant, occurring in two or more cell regions.
    We also showed that the Internet's physical topology graph as a whole is strongly connected, with a minimum of 986 regions requiring complete obliteration to fully halve the Internet, with half of all prefixes in a partition leaving them fully unable to communicate with any prefix in the other partition.
    However, regional partitioning remains a threat, as less well-connected parts of the world direct most of their traffic through a limited number of peering regions. \shaddi{examples}
