Our work builds on and takes inspiration from research in Internet resilience
analysis, experience from Internet outages, PoP-level Internet topology
measurement studies, and network infrastructure security policy.

{\bf Resilience Analysis.}
     Comprehensive analysis of Internet resilience remains beyond reach due to
    limited access to global routing and topology data.  However, limited
    studies provide insight into the impact of logical link failures and
    opportunities to make the AS graph more robust to these failures.
    
    Wu et. al~\cite{michigan} provide the most in-depth analysis of Internet
    resilience under {\it logical} link failure.  Using an AS-level graph, they
    removed one or more peering relationships and evaluated the availability of
    policy-compliant paths between impacted networks before and after the
    logical link failure.

    Omer et. al~\cite{measuringresilience} consider undersea cables as capacity
    bottlenecks for Internet connectivity between continents.  Like us, they
    look for constraints in the physical, rather than logical connectivity
    graph.  However, their model focuses on inter-continental connectivity
    rather and does not focus on IXPs as correlated failure points.

    \eat{
    Albert
    et al.~\cite{resilience-complex-networks} studied two different complex
    networks: Erd\"{o}s-R\'{e}nyi (ER) and scale-free. An ER model produces a
    network with an exponential tail and a scale-free has a power-law tail for
    node degree. Against random node removal the scale-free model had a high
    degree of tolerance, while the ER model was not. However for a deliberate
    attack of removing the most connected nodes the scale-free model quickly
    deteriorates and the ER model is effected the same as if they were
    randomly removed. This work is different from ours since we created our
    models from actual Internet measurement, which do not necessarily exhibit
    the power law.}
    %After reading this, let's make sure to also try to come to some
    %conclusions about the properties of the graph we create.  Note: the
    %Internet does not exhibit the power law!
    \shaddi{I don't understand how this is relevant to our work, particularly
    as the degree of nodes in the Internet is known not to follow an ER /or/
    scale-free model (~\cite{first-principles-router-topology}, section
    2). Am I missing something? Or is this just for background information?}
    \justine{agreed - good to have read it to make sure we weren't missing something but it doesn't sound relevant to the related work. I ate the text so we can put it back if we're missing something.}

    Dolev et al.~\cite{resilience-under-BGP} studied resilience over an AS connectivity graph annotated with
    on BGP policy relationships. They found that the Internet
    is much more susceptible to 
 deliberate attacks of removing the AS's with
    highest degree than previous studies that did not take into consideration
    BGP polices. Our work is different from Dolev et al.'s because we look at
    the physical failure points of the Internet, rather than the logical
    AS-level graph.
    \justine{Were they before or after the Michigan CoNext paper? Because the Mich. paper included nice metrics and considered policy, too}
    %Their metrics might be useful for us: average shortest path length, size
    %of largest component, number of connected node pairs in the network,
    %ratio between pairs connected and all pairs
    
    Hu et. al~\cite{ixp-routingdiversity} also investigate logical failures on
    an AS-level graph, focusing opportunities to limit the impact of logical
    failures.
    %They looked at relaxing peering constraints (relax valley-freeness)

    Like us, they augmented thair AS graph with IXP connectivity data.  Unlike
    our work, they did not use this data to consider the possibility of
    correlated failures.  Rather, they investigated using IXPs to provide
    backup peers to improve connectivity in case of emergency.

{\bf Outage Events.}  Real outage events highlight the threat of correlated
    failure in geographic locations which suffer disaster or attack.  For
    example, the 2006 Boxing Day Earthquake in Taiwan resulted in major
    Internet outages when six of seven undersea cables connecting North and
    Southeast Asia with eachother and North America
    disconnected~\cite{asia-comm-quake}.  While the earthquake did not cause
    complete disconnectivity, the affected networks suffered heavy congestion
    until the cables were repaired.  This experience served operators as a case
    study for emergency restoration in case of disaster~\cite{taiwan}.
       
    \eat{ 
    \justine{agreed wrt the 2003 study, doesn't look as relevant as the Taiwan
    quake}}
    %\item Restoration study~\cite{taiwan}

    %\item 2003 Blackout in Northeastern US and Canada~\cite{blackout2003}.
    %Cowie et al. Looked at the prefixes that were effected by the blackout.
    %They found that even though the Internet backbone providers were
    %unaffected by the blackout many businesses were offline for hours to
    %days, which affected many.
    %Note sure if this report will be very useful, only looked at connectivity
    %in the geographical area of the blackout, not how the blackout affected
    %the general Internet traffic patterns and such.

    %After the Taiwan earthquake in 2006 the Internet was resilient, but
    %congested~\cite{asia-comm-quake}. The Internet was able to recover and
    %switch to back-ups relatively quickly. However the loss of service can
    %still be catastrophic for both ISP's and businesses. Businesses lose
    %information needed to run their business, while ISP's lose traffic due to
    %congestion or must switch to backup links that are usually more expensive
    %without compensation.

{\bf PoP-level Internet Topologies.}  Our model for network connectivity
    focuses on {\it failure points}, physical locations where multiple
    networks connect, leading to multiple correlated failures in case of
    disaster.  We borrow techniques and data for discovering these
    physical locations from iPlane~\cite{iplane} and the IXP Mapping
    Project~\cite{ixps-mapped}.  iPlane clusters IP addresses into PoPs
    using a combination of DNS-based geolocation and TTL-based distance
    measurements.  The IXP Mapping project uses public IXP membership
    datasets, DNS names, looking glass servers, BGP tables, active
    traceroute and ping measurements, and other sources to provide the
    most accurate IXP membership datasets available to date. Cai et al.  propose
    an organization-based model of the Internet that takes into account the fact
    that a single organization may control multiple ASes~\cite{as-to-org}. While
    our analysis here treats each AS as an individual organization, the
    techniques we use easily generalize to a topology model that accurately
    incorporates these relationships. \shaddi{I think this means that our model
    /underestimates/ the degree of resilience, but that's going to depend on
    what our graph looks like.I'd imagine that there are going to be missing AS
    links in our model that will cause us to violate our assumption that each
    administrative entity (organization) is well connected and cannot be
    partitioned.}
    \justine{Agreed that the Cai work enhances AS level graphs, but (1) the graphs that we have do take in to account AS `sibling' relationships, and (2) I don't think we're planning on focusing on policy? Or are we? Either way I think (1) means we should cut the Cai ref.}

{\bf Network Security Policy.}
    \justine{cut this section unless we find something more compelling?}
    \begin{itemize}
        \item White House POlicy Review~\cite{cyberspacepolicy} 
    \end{itemize}
