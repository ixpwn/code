
In order to study the impact of regional disaster, we augment a standard AS
logical connectivity graph~\cite{caida-asgraph} with measurements of the
geographic regions in which networks peer.  We construct a grid of roughly
equal-area hexagonal cells which tile the Earth's surface, and map latitude and
longitude coordinates to a cell in the grid.  We consider each cell of the globe
as a `region,' such that a worst-case disaster could destroy all of the physical
routing infrastructure within the region.  This results in an Icosahedral  
a standard algorithm~\cite{geodesic}, providing us with 81,000 triangular
faces, each with average side length of 100km and area 4700 km$^2$.  Finally,
we map locations where ASes peer to faces on the globe.  The same technique
could be applied to intradomain connectivity for large networks, but we leave
this analysis to future work.    

We develop our peering dataset by borrowing techniques and data from the IXP
Mapping Project~\cite{ixps-mapped}, iPlane~\cite{iplane},
Aqualab~\cite{sidewalk}, CAIDA~\cite{caidadata}, and the MaxMind~\cite{maxmind}
industrial IP geolocation service.  For the iPlane and Aqualab data, we derive
peering points from raw traceroute files.  For each pair of IP addresses we
observe adjacent in a traceroute, we map the IPs to alias clusters generated by
CAIDA's iffinder~\cite{iffinder} and MIDAR~\cite{iffinder, midar} measurements.
We then map the alias clusters to ASes, and if the ASes are different, we store
the two IP addresses as an AS peering.  Since the majority of stub networks are
not geographically widespread, we limited our study to networks which provided
transit to at least one AS; thus, we removed any adjacency which included a
stub network.  We map the peering to a geographic location using a combination
of DNS-based geolocation and the MaxMind commercial geolocation service.  For
both IP addresses in the adjacency, we do reverse DNS lookups on all IP
addresses in their alias clusters.  If any of the DNS names match rules from
the undns~\cite{undns} or sarang~\cite{sarang} projects, we map the IP address
to that location.  If multiple of the IP addresses point to different
locations, we elect the most common location, or in a tie, choose the location
with the least distance from the others.  If no DNS names are found, we repeat
the same process with locations from the MaxMind GeoLite City database, which
has better coverage that DNS geolocation, but worse
accuracy~\cite{uhlig_ccr_paper}. 

The measurements we gathered from CAIDA, iPlane, and Aqualab allowed us to
identify 489,334 router-level AS adjacencies, which we mapped to 14,457 unique
latitude, longitude locations (133,686 using DNS, and the remainder using
MaxMind).  We then supplemented these measured adjacencies with 38,994
adjacencies from a ground-truth dataset~\cite{ixps-mapped}, bringing our total
to 528,328 adjacencies in 14,571 uniqe latitude, longitude locations.  We
mapped these adjacencies to 2,900 unique faces of the geodesic globe.   

 
\subsubsection*{Evaluation of Model}

    To evaluate the representativeness of our data, we investigate the fraction
    of logically connected ASes for whch we were able to identify geolocated
    adjecencies, compare our measured adjacencies to a set of ground truth
    adjacencies, and perform a survey of network operators to validate our
    data.

    {\bf Representation of Logical Connectivity.} Finite vantage points,
    policy-based routing, backup links, and limited geolocation capabilities
    limit our ability to exaustively discover AS adjacencies.  Policy-based
    routing means that traceroutes issued from within a network, or from a
    network's customer AS, may be able to traverse paths that an issuer from
    outside the network may not be able to traverse.  With only finite vantage
    points limited to a fraction of networks, our measurements may fail to
    identify many AS-level links.  Further, backup links between ASes may never
    be traversed by our traceroute measurements, since backup links are by
    definition offline unless a failure occurs on a primary link.  Finally,
    since geolocation (especially for routers rather than end hosts) is
    limited, even when we identify an AS-level adjacency we may not be able to
    geolocate it.
    
    As a first step towards identifying how representative our dataset is, we
    count how many AS peerings our geolocated adjacencies capture vs those from
    more exhaustive datasets (note that ground truth data is unavailable).
    \justine{TODO: this.}

    {\bf Ground Truth Adjacencies.}
\begin{figure}[tb]
\centering
\includegraphics[width=3.25in]{graph_all_match}
\caption[]{\label{fig:closestadjacency} For ground truth peering locations, CCF
of distance (kilometers) to closest AS adjacency in our measurement-based
dataset. X-axis is in log scale. The average ground truth adjacency is within
100km of a measured adjacency, placing it in either in the same face or a face
adjacent to its ground-truth location.} 
\end{figure}
    We next take a set of `ground-truth' adjacencies, peerings with known
    locations, and compare them to our measured adjacencies.  The ground truth
    adjacencies are borrowed from the IXP Mapping Project~\cite{ixps-mapped},
    which manually investigated every public Internet Exchange Point worldwide.
    For each adjacency, we mapped it to the measured adjacency nearest to it on
    the globe.  Figure~\cite{fig:closestadjacency} shows the results in CDF
    form.  We provide results for comparison to measured adjacencies geolocated
    using our combined techinique (DNS and Maxmind geolocations both used), DNS
    geolocation only, MaxMind geolocation only, and strawman geolocation in
    which all adjacencies are mapped to Greenwich, England.  \justine{TODO:
    describe} 




    {\bf Operator Survey.} {\it Due to our poor performance on the previous two
    metrics, we decided to hold off on our operator survey until we have data
    we feel confident in sending out for operator validation.}



